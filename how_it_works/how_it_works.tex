\documentclass{article}

\usepackage{graphicx}
\usepackage{amsmath}
\usepackage{braket}
\usepackage{float}
\usepackage[margin=1in]{geometry}


\def\hwtitle{How the Transients Simulations Work}
\def\hwauthor{Sarah Chastain}
%\def\hwdate{Jan. 24, 2020}

\usepackage{fancyhdr}
\lhead{\hwauthor}
\chead{\hwtitle}
\rhead{\date}
\lfoot{\hwauthor}
\cfoot{\today}
\rfoot{\thepage}
\renewcommand{\footrulewidth}{0.4pt}
\pagestyle{fancy}
%
\author{\hwauthor}
\title{\hwtitle}
\date{\today}

\begin{document}
	
	\maketitle
	\thispagestyle{fancy}
	
	
\section{Input and Configuration}
Before running the code it is necessary to provide some input in terms of both the observation set up and in the simulation parameters. For this purpose, the code reads in one or perhaps two files:
\subsection{config.ini}
This file is required. It contains two sections.
\subsubsection{Initial Parameters}
These are the parameters to be used for the simulations themselves. See the commented file for more details of what each parameter is. 
\subsubsection{Simulation Parameters (Sim)}
If there is no provided file with observation details, then this section contains the observation parameters.
\subsection{Observation file}
This file is optional. It contains the details of the observations in a list with columns.

\section{Parsing the Input}
The code then parses the input supplied from the command line and the input files listed above. The command line options include the option to only do the simulation calculations, to only plot the results, and to save all of the intermediate steps.
\section{Observation Parameters}
 The code then converts the inputted observation parameters into a usable format for the rest of the code in the ``observing\_strategy'' function. 
 \section{Generating Simulated Sources}
 Then, provided the users desires the simulations be calculated, this is passed on to a function that generates simulated sources. 
 \subsection{Durations}
 The first step is to generate durations that are randomly selected from a logarithmically uniformly spaced distribution of durations from the specified minimum to the specified maximum.
 \subsection{Fluxes}
 Next, the fluxes are simulated, yet again, from a logarithmically uniformly spaced distribution of fluxes from the specified minimum to the specified maximum.
 \subsection{Start Times}
 The sources are given a randomized start time that depends on the duration along with some of the observation parameters. Since the beginnings of these light curves are immediate in time, the only transients that will be detected are those which start before the end of the observation period. However, the transients may start before the beginning of the first observation. As far back as one transient duration before the first observation began. Therefore, the start time will be randomly chosen to be between one duration before the start of the observations and the end of observations.
\section{Detection of Simulated Sources}
These simulated sources are then run through a function that determines whether they are detected in the observations. The idea is to iterate over the observations and in each observation look to see what sources are detected. We follow the section criteria in the following sections.
\subsection{Edge Cases}
The first step in determining which simulated sources are candidates. The three implemented light curve types have different approaches:
\subsubsection{FRED or Half-Gaussian Type 1}
The start time of the sources must not be equal to the end of the observation.
\subsubsection{Wilma}
The transients cannot end before the observation starts.
\subsubsection{Tophat/Parabolic/Chopped Gaussian/ERED}
There are two conditions: the source must not start and end before or right at the start of observations and the source must not start at the end of the observation.
\subsubsection{Gaussian}
We do no additional conditions here since this light curve has already had sufficient conditions imposed.
\subsection{Integrated Flux Filtering}
We now add some randomized errors to the peak flux that we simulated. Then we integrate over the observation to find the integrated flux, the specifics of which are outlined below. The integrated fluxes are then tested to see if they exceed the sensitivity limit or the optional extra sensitivity limit. 
\subsubsection{FRED}
The flux of a FRED lightcurve is given by:
\[f(t)=f_0e^{-t/\tau}\]
Where $f_0$ is peak flux, $t$ is time, $\tau$ is duration. We seek to integrate over the observation time:
\[f_{int}(T_{end}-T_{start}) = f_0\int_{t_{start}}^{t_{end}}e^{-t/\tau}dt\]
\[t_{start} = \text{Max}(t_{burst}, T_{start}) - t_{burst}\]
\[t_{end} = T_{end} - t_{burst}\]
Where $T_{start}$ is the start  of the observation , $T_{end}$ is the end of the observation, and $t_{burst}$ is the start of the transient. We have the duration of observation term on the left hand side since we are actually calculating the fluence with this integration. We calculate the integral to find:
\[f_{int}(T_{end}-T_{start}) = f_0\tau (e^{-t_{start}/\tau}-e^{-t_{end}/\tau})\]
Simply solving for the integrated flux:
\[f_{int} = f_0\tau \frac{e^{-t_{start}/\tau}-e^{-t_{end}/\tau}}{T_{end}-T_{start}}\]
\subsubsection{Wilma}
The flux of a Wilma lightcurve is like the FRED in reverse. 
\[f(t)=f_0e^{t/\tau}\text{, for }t\leq0\]
We seek to integrate over the observation time:
\[f_{int}(T_{end}-T_{start}) = f_0\int_{t_{start}}^{t_{end}}e^{t/\tau}dt\]
\[t_{start} =  T_{start} - (t_{burst}+\tau)\]
\[t_{end} = \text{Min}(T_{end},t_{burst}+\tau) - (t_{burst}+\tau)\]
\[f_{int}(T_{end}-T_{start}) = f_0\tau (e^{t_{end}/\tau}-e^{t_{start}/\tau})\]
Simply solving for the integrated flux:
\[f_{int} = f_0\tau \frac{e^{t_{end}/\tau}-e^{t_{start}/\tau}}{T_{end}-T_{start}}\]
\subsubsection{ERED}
For the Exponential Rise-Exponential Decay we have:
\[f(t)=f_0e^{t/\tau}\text{, for }t<0\]
\[f(t)=f_0e^{-t/\tau}\text{, for }t>0\]
We just reuse the previous two sections and take $\tau = \tau/2$. Starting with the FRED:
\[t_{start} = \text{Max}(t_{burst}, T_{start}) - t_{burst}\]
\[t_{end} = T_{end} - t_{burst}\]
\[f_{int, FRED} = f_0\tau \frac{e^{-t_{start}/\tau}-e^{-t_{end}/\tau}}{T_{end}-T_{start}}\]
Now the Wilma:
\[t_{start} =  T_{start} - (t_{burst}+\tau)\]
\[t_{end} = \text{Min}(T_{end},t_{burst}+\tau) - (t_{burst}+\tau)\]
\[f_{int, Wilma} = f_0\tau \frac{e^{t_{end}/\tau}-e^{t_{start}/\tau}}{T_{end}-T_{start}}\]
\[f_{int} = f_{int, FRED} + f_{int, Wilma}\]
\subsubsection{Tophat}
For this type of light curve, the flux is:
\[f(t) = f_0\]
Therefore we once again start by calculating the fluence:
\[f_{int} (T_{end}-T_{start}) = f_0 (t_{end}-t_{start})\]
\[t_{start} =  \text{Max}(T_{start},t_{burst}) - t_{burst}\]
\[t_{end} = \text{Min}(T_{end},t_{burst}+\tau) - t_{burst}\]
\[f_{int}  = f_0 \frac{t_{end}-t_{start}}{T_{end}-T_{start}}\]
\subsubsection{Parabolic}
For this type of light curve the flux is:
\[f(t) = -\frac{f_0}{(\tau/2)^2}(t-\tau/2.0)^2 + f_0\]
We start by calculating the fluence:
\[f_{int}(T_{end}-T_{start}) = f_0(t_{end} - t_{start}) - \frac{f_0}{3(\tau/2)^2}[(t_{end}-\tau/2)^3 - (t_{start}- \tau/2)^3]\]
\[t_{start} =  \text{Max}(T_{start},t_{burst}) - t_{burst}\]
\[t_{end} = \text{Min}(T_{end},t_{burst}+\tau) - t_{burst}\]
Therefore, we have:
\[f_{int} = f_0\frac{t_{end} - t_{start}}{T_{end}-T_{start}} - \frac{f_0}{3(\tau/2)^2(T_{end}-T_{start})}[(t_{end}-\tau/2)^3 - (t_{start}- \tau/2)^3]\]
\subsubsection{Gaussian}
For the Gaussian light curve we have a flux of:
\[f(t) = f_0  \exp[-\frac{(t-(t_{burst}+\frac{\tau}{2}))^2}{2(\frac{\tau}{x})^2}]\]
In this formulation, $x$ is the number that determines the standard deviation from the duration of the transient. The duration is a tricky concept for a function that is never truly reaches zero; therefore, it is defined instead by the point at which the flux rises above the sensitivity of the observation. We calculate this by setting the flux equal to the sensitivity of the observation and setting the time equal to $t_{burst}$:
\[S_{obs} = f_0 \exp[-\frac{(t_{burst}-(t_{burst}+\frac{\tau}{2}))^2}{2(\frac{\tau}{x})^2}]\]
\[S_{obs} = f_0 \exp[-\frac{(\frac{\tau}{2})^2}{2(\frac{\tau}{x})^2}]\]
Simplifying:
\[S_{obs} = f_0 \exp[-\frac{x^2}{8}]\]
Solve for $x$:
\[x=\sqrt{-8\ln(\frac{S_{obs}}{f_0})}\text{ if }f_0>S_{obs}\]
If $f_0<S_{obs}$, then the transient will not be detected. (This is handled in the code by setting $x=1$ and letting it get rejected later).

Now, note the typical Gaussian used in probability:
\[ \frac{1}{\sqrt{2\pi\sigma^2}}\exp[-\frac{(x-\mu)^2}{2\sigma^2}]\]
The differences between the Gaussian light curve and the regular Gaussian come from a few physical conditions: Firstly, the flux is defined such that at the mean the flux is exactly the peak flux. Second, the ``duration'' (a slightly more complicated concept for this light curve) is defined based on the sensitivity of the observation and the duration of the source as shown above. Therefore from this definition we decide where to place the mean. A logical choice is to be the simulated beginning plus half the duration. 

As done previously we can now integrate to calculate fluence and subsequently the integrated flux. 
\[f_{int}(T_{end}-T_{start}) = f_0 \int_{T_{start}}^{T_{end}} \exp[-\frac{(t-(t_{burst}+\frac{\tau}{2}))^2}{2(\frac{\tau}{x})^2}]dt\]
$T_{start}$ is the start of the observation and $T_{end}$ is the end of the observation. Integrating this equation gives:
\[f_{int}(T_{end}-T_{start}) = f_0 \frac{\tau}{x}\sqrt{\frac{\pi}{2}}(\text{erf}(\frac{T_{end}-(t_{burst}-\frac{\tau}{2})}{\sqrt{2}\frac{\tau}{x}})-\text{erf}(\frac{T_{start}-(t_{burst}-\frac{\tau}{2})}{\sqrt{2}\frac{\tau}{x}}))\]
Compare this with the Gaussian cumulative distribution function:
\[F_{cdf}=\frac{1}{2}(1+\text{erf}(\frac{x-\mu}{\sqrt{2}\sigma}))\]
Therefore, to make them look similar:
\[2F_{cdf}=(1+\text{erf}(\frac{x-\mu}{\sqrt{2}\sigma}))\]
Use our variables:
\[2\Delta F_{cdf}=(1+\text{erf}(\frac{T_{end}-(t_{burst}-\frac{\tau}{2})}{\sqrt{2}\frac{\tau}{x}}))-(1+\text{erf}(\frac{T_{start}-(t_{burst}-\frac{\tau}{2})}{\sqrt{2}\frac{\tau}{x}}))\]
Simplify:
\[2\Delta F_{cdf}=(\text{erf}(\frac{T_{end}-(t_{burst}-\frac{\tau}{2})}{\sqrt{2}\frac{\tau}{x}}) - \text{erf}(\frac{T_{start}-(t_{burst}-\frac{\tau}{2})}{\sqrt{2}\frac{\tau}{x}}))\]
Therefore we can write the integrated flux in terms of this cumulative distribution function:
\[f_{int}(T_{end}-T_{start}) = f_0 \frac{\tau}{x}\sqrt{\frac{\pi}{2}}2\Delta F_{CDF}\]
Simplify and solve for integrated flux:
\[f_{int} = f_0 \frac{\tau}{x(T_{end}-T_{start})}\sqrt{2\pi}\Delta F_{CDF}\]
\subsubsection{Chopped Gaussian}
For the Chopped Gaussian, we are looking at a Gaussian light curve, but only between a certain number of standard deviations of each light curve. Let us label this chopping parameter $c$.
Therefore we just have the Gaussian light curve:
\[f(t) = f_0  \exp[-\frac{(t-(t_{burst}+\frac{\tau}{2}))^2}{2(\frac{\tau}{c})^2}]\]
With the integrated flux:
\[f_{int} = f_0 \frac{\tau}{c(T_{end}-T_{start})}\sqrt{2\pi}\Delta F_{CDF}\]
or also:
\[f_{int} = \frac{f_0}{T_{end}-T_{start}} \frac{\tau}{c}\sqrt{\frac{\pi}{2}}(\text{erf}(\frac{T_{end}-(t_{burst}-\frac{\tau}{2})}{\sqrt{2}\frac{\tau}{c}})-\text{erf}(\frac{T_{start}-(t_{burst}-\frac{\tau}{2})}{\sqrt{2}\frac{\tau}{c}}))\]
\subsubsection{Half Gaussian Type 1}
For this light curve we consider a light curve that starts with the peak flux and falls off as a Gaussian. In order to implement this light curve we again take the Gaussian and modify it:
\[f(t) = f_0  \exp[-\frac{(t-t_{burst})^2}{2(\frac{\tau}{x})^2}]\text{ for }t>t_{burst}\]
Therefore the integrated flux:
\[f_{int} = \frac{f_0}{T_{end}-T_{start}} \frac{2\tau}{x}\sqrt{\frac{\pi}{2}}(\text{erf}(\frac{T_{end}-t_{burst}}{\sqrt{2}\frac{2\tau}{x}})-\text{erf}(\frac{\text{Max}(T_{start},t_{burst})-t_{burst}}{\sqrt{2}\frac{2\tau}{x}}))\]
Note we changed the standard deviation to be $2\tau/x$ in order to account for only having half of the gaussian. Also, we changed the mean to be at the start of the transient and changed the start of the integration be at either the start of the observation or the start of the transient. Whichever is later.
\subsection{Unique Detections}
The next step is to ensure that each transient that is detected is marked as a transient multiple times. After all of the loops over the observations, the list of transients is run through a function that marks which detections are unique and only the unique detections are kept.

\section{Detection Statistics}
After simulating and detecting the transients the next step is to calculate the detection statistics. 

\subsection{Detection Intervals}
The first step is to divide the flux and duration axes into logarithmic intervals that are divided into 20 intervals per order of magnitude. 
\subsection{Loop over Intervals}
The next step is to loop over the intervals and bin the durations of the transients. After calculating a duration bin, the flux bins are calculated. This is looped through for each duration bin since part of the flux bin calculation depends on the detected duration statistics. 
\subsection{Calculate Probabilities and Print}
The probabilities are then calculated based on the ratio of detections to simulated sources. The output is printed to a file.

\section{Plotting Results Including Lines}
The file is then read back in so that it can be plotted as a contour plot. None of the plotting routines in python handle log-log contour plots very well so there is a lot of np.log10 statements followed by 10**, but in the end it all works out and can be verified by plotting in Mathematica if one prefers. 
\subsection{Surface Plot}
The surface plot itself is generated based on the probabilities calculated in the previous section. In python this is accomplished using the np.meshgrid function. 

\subsection{Horizontal Lines}
There are three lines on the plots overplotted: one is the minimum sensitivity of all observations, one is the maximum sensitivity of the observations, and one is the maximum sensitivity with the optional extra threshold applied.

\subsection{Vertical Lines}
The duration of the first observation is plotted as a vertical line for each kind of light curve. Each different kind of light curve has two other kinds of lines plotted.

\subsubsection{FRED}
We start be establishing a limit that should determine a boundary between almost certain detections and less certain detections:
\[f_{int} = S_{obs}\]
Therefore, we can use the previously calculated formula for integrated flux:
\[f_0\tau \frac{e^{-t_{8}/\tau}-e^{-t_{9}/\tau}}{T_{end}-T_{start}} = S_{obs}\]
Now the only difficult part is determining what $t_8$ and $t_9$ are. Consider a bright transient that might fall into one of two different scenarios:
\begin{itemize}
	\item It is bright enough to almost be detected from the beginning of the first observation until the last observation \begin{itemize}
		\item $t_8$ is the time of the end of observations, $t_{end}$; minus the start of the first observation, $t_{start}$; minus the duration of an observation, $(T_{end}-T_{start})$; plus the duration, $\tau$
		\item $t_9$ is the time of the end of observations,  $t_{end}$; minus the start of the first observation, $t_{start}$; plus the duration
		\end{itemize}
	\item It is bright enough to be detected from the start of observations, but short enough to perhaps fall in the gap before the next observation\begin{itemize}
		\item $t_8$ is the duration of a gap, $\tau_{gap}$; plus the duration of a observation, $(T_{end}-T_{start})$
		\item $t_9$ is the duration of a gap, $\tau_{gap}$; plus the duration of a observation, $(T_{end}-T_{start})$; plus the duration of another observation, $(T_{end}-T_{start})$
	\end{itemize} 
	\end{itemize}

In the first scenario we have:
\[f_0(\tau) = S_{obs}\frac{\tau}{T_{end}-T_{start}}[\exp(\frac{-(t_{\text{end}}-t_{\text{start}}-(T_{end}-T_{start})+ \tau )}{ \tau})-\exp(\frac{-(t_{\text{end}} - t_{\text{start}} + \tau)}{\tau})]^{-1}  \]
In the second scenario we have:
\[f_0(\tau) = S_{obs}\frac{\tau}{T_{end}-T_{start}}[\exp(\frac{-(\tau_{gap} + (T_{end}-T_{start}))}{ \tau})-\exp(\frac{-(\tau_{gap} + 2(T_{end}-T_{start}))}{\tau})]^{-1}\]

\begin{figure}[H] 
	\begin{center}
		\includegraphics[width=4in]{output_fred_ProbContour.png}
				
		\label{FRED}
	\end{center}
\end{figure}
\subsubsection{Wilma}
\subsubsection{ERED}
\subsubsection{Tophat}
One of the lines plotted is the longest duration that a transient could have and still fall into a gap in the observation intervals. This corresponds to the length of time just after the first observation and just before the last observation. The other line is the longest possible duration a source could have before being detected as a constant source. This corresponds to the duration from the beginning of the first observation until the end of the last observation.


\begin{figure}[H] 
	\begin{center}
		\includegraphics[width=4in]{output_tophat_ProbContour.png}
		
		\label{tophat}
	\end{center}
\end{figure}
\subsubsection{Parabolic}
\subsubsection{Gaussian}
Once again we enforce a limit that guides us to limiting cases:
\[f_{int} = S_{obs}\]
 First, we use the equation for integrated flux we had previously:
\[x_{max}=\sqrt{-8\ln(\frac{S_{obs}}{f_{max}})}\text{ if }f_0>S_{obs}\]
\[f_0 \frac{\tau}{x_{max}(T_{end}-T_{start})}\sqrt{2\pi}\Delta F_{CDF} =  S_{obs}\]
Inserting the variables:
\[f_0 \frac{\tau}{x_{max}(T_{end} - T_{start})}\sqrt{2\pi}(F_{CDF}(t_8; \tau/2,\tau/10) -F_{CDF}(t_9; \tau/2, \tau/10)) =  S_{obs}\]

We repeat the same scenarios as outlined above in the FRED case:
\begin{itemize}
	\item It is bright enough to almost be detected from the beginning of the first observation until the last observation \begin{itemize}
		\item $t_8$ is the time of the end of observations, $t_{end}$; minus the start of the first observation, $t_{start}$; minus the duration of an observation, $(T_{end}-T_{start})$; plus the duration, $\tau$
		\item $t_9$ is the time of the end of observations,  $t_{end}$; minus the start of the first observation, $t_{start}$; plus the duration
	\end{itemize}
	\item It is bright enough to be detected from the start of observations, but short enough to perhaps fall in the gap before the next observation\begin{itemize}
		\item $t_8$ is the duration of a gap, $\tau_{gap}$; plus the duration of a observation, $(T_{end}-T_{start})$
		\item $t_9$ is the duration of a gap, $\tau_{gap}$; plus the duration of a observation, $(T_{end}-T_{start})$; plus the duration of another observation, $(T_{end}-T_{start})$
	\end{itemize} 
\end{itemize}
For the first case we have:
\[f_0  =  S_{obs}\frac{x_{max}(T_{end} - T_{start})}{\sqrt{2\pi}\tau}(F_{CDF}(t_{\text{end}}-t_{\text{start}}-(T_{end}-T_{start})+ \tau; \tau/2,\tau/x_{max}) -F_{CDF}(t_{\text{end}} - t_{\text{start}} + \tau; \tau/2, \tau/x_{max}))^{-1}\]
For the second case we have:
\[f_0  =  S_{obs}\frac{x_{max}(T_{end} - T_{start})}{\sqrt{2\pi}\tau}(F_{CDF}(\tau_{gap} + (T_{end}-T_{start}); \tau/2,\tau/x_{max}) -F_{CDF}(t_{\text{end}} - t_{\tau_{gap}} + 2(T_{end}-T_{start}); \tau/2, \tau/x_{max}))^{-1}\]
In the figure below, we see only part of the line from the second case, but it matches nicely. 
\begin{figure}[H] 
	\begin{center}
		\includegraphics[width=4in]{output_gaussian_ProbContour.png}
		
		\label{gaussian}
	\end{center}
\end{figure}
\subsubsection{Chopped Gaussian}
\subsubsection{Half Gaussian Type 1}
\end{document}
